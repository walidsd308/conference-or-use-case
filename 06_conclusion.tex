\section{Conclusion}

This patient-centered operating room (OR) scheduling framework balances operational optimization with patient autonomy and shared decision-making. Implemented in the orthopedic department at CHU Montpellier (France), the system demonstrates computational feasibility and operational benefits of patient-centered surgical scheduling.

The solution uses a mixed-integer linear program with hierarchical objectives, generating up to three distinct scheduling options per patient. A Flexibility-First heuristic prioritizes flexible patients (availability score $\sigma_p$) for early selection, preserving options for constrained patients. Patients actively participate in their care timeline.

Benchmark simulations across 10--40 patients, 1--2 physicians, and 2--3 operating rooms demonstrate performance across diverse scenarios. Real-world validation (134 patients, 3 weeks) achieved 72--73\% to 91--95\% utilization increases and 36 net operations gained versus 11 in the real case. For scales $P \le 20$, solutions arrive in under 18 seconds, enabling real-time workflow. The weighted hierarchical objective balances OR capacity with shared decision-making.

The framework was developed for elective orthopedic surgery, where scheduling flexibility is higher. Validation in other surgical specialties and at larger institutional scales is necessary.

Future work will address stochastic elements, machine learning for preferences, and real-time rescheduling. Genetic algorithms may enhance scalability for larger institutional deployments.

Operational excellence and patient care integrate through thoughtful optimization design, demonstrating that patient preferences enhance healthcare operations management.
