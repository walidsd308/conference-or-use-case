\section{Numerical Results}\label{sec:numerical_results}
\subsection{Hospital Data Structure}
The computational experiments use data from the Department of Orthopedic Surgery at CHU Montpellier for elective surgical procedures over a three-week period. This framework is used to test the model's ability to optimize and condense the pre-existing surgical schedule.

\subsubsection{Patient and Procedure Data}

The dataset includes $134$ patients requiring surgery, categorized into four main procedure types: arthroplasty, foot and ankle surgery, soft tissue repair, and  arthroscopy.

Operative time ranges from 15 to 240 minutes, with mean duration 71.86 minutes and median 80 minutes. A fixed $50$-minute post-operative turnover time is assumed for all cases.

\subsubsection{Resource and Block Constraints}

A total of $11$ physicians are involved. Their schedules are fixed, requiring availability in at least one of the two weeks modeled. Two head physicians are included; one is capable of managing "swinging rooms," operating concurrently in up to $3$ Operating Rooms (ORs).

Three OR blocks are available all working weekdays, providing $10.5$ hours of capacity daily. One limited OR is open only on Wednesdays and Fridays, providing $6.5$ hours of capacity daily.

\subsection{Performance Metrics on Real-World Data}

The results, summarized in Table \ref{tab:utilization} and Figure \ref{fig:utilization}, demonstrate the model's capacity to improve resource deployment while maintaining operational flexibility. To isolate maximum resource utilization, tests limited each patient to a single scheduled event. This step directly assessed the theoretical capacity within existing real-world resource constraints.
\begin{table}[htb]
\centering
\caption{Comparison of Surgical Block Time Utilization (\%): Real Case vs. Optimized Schedule over 3 Weeks}
\label{tab:utilization}
\begin{tabular}{cccc}
\toprule
\textbf{Week} & \textbf{Real Case (\%)} & \textbf{Optimized (\%)} & \textbf{Change (p.p.)} \\
\midrule
1 & 72\% & 95\% & +23 \\
2 & 73\% & 91\% & +18 \\
3 & 79\% & 39\% & -40 \\
\bottomrule
\end{tabular}
\end{table}

\begin{figure}[htb]
    \centering
    \begingroup
        \includegraphics[scale=.38]{figures/figure_graph_per.pdf}
    \endgroup
    \caption{Surgical Block Time Utilization: Real Case vs. Optimized Schedule.}
    \label{fig:utilization}
\end{figure}
% \begin{table}[htb]
% \centering
% \caption{Comparison of Net Operations Gained: Real Case vs. Optimized Schedule over 3 Weeks}
% \label{tab:operations}
% \begin{tabular}{ccc}
% \toprule
% \textbf{Week} & \textbf{Real Case (Operations)} & \textbf{Optimized (Operations)} \\
% \midrule
% 1 & 18 & 3 \\
% 2 & 16 & 5 \\
% 3 & 11 & 36 \\
% \bottomrule
% \end{tabular}
% \end{table}

% \begin{figure}[htb]
%     \centering
%     \begingroup
%         \includegraphics[scale=0.38]{figures/figure_graph_count.pdf}
%     \endgroup
%     \caption{Net Operations Gained: Real Case vs. Optimized Schedule.}
%     \label{fig:operations}
% \end{figure}

\noindent
The optimization framework demonstrates strategic redistribution across the planning horizon. In Weeks 1--2, utilization increased from 72--73\% to 91--95\% by condensing workload into available block time. Week 3 showed reduced utilization (79\% to 39\%) as capacity was maximized early; however, overall throughput was superior: the optimized schedule achieved 36 net operations gained versus 11 in the real case over the three-week horizon.
\subsection{Simulation Results and Benchmark Design Rationale}

The benchmark design reflects the operational reality of the deployed system. Unlike traditional OR scheduling approaches that optimize large patient batches at fixed intervals (e.g., weekly master surgical schedules), this framework operates in consultation-driven, incremental mode.
\subsubsection{Operational Context Justifying Benchmark Scale}
The scheduling process integrates with consultation: patients enter immediately after consultation, are assigned to their surgeon's schedule, and generate 3 feasible time slots within seconds. The system processes patients continuously, maintaining physician wait lists of 5--20 patients. This continuous-flow model prevents backlogs, making benchmark instances of 10--40 patients realistic operational scenarios (Table \ref{tab:benchmark_data}).


\begin{table}[ht]
    \centering
    \caption{Benchmark Data Results}
    \label{tab:benchmark_data}
    \resizebox{\columnwidth}{!}{%
        \begin{tabular}{ccccccc}
            \toprule
            Physicians & Patients & Blocs & Events & \parbox{1.5cm}{\centering Patients\\with Events} & \parbox{1.2cm}{\centering Avg per\\Patient} & \parbox{1.2cm}{\centering Run\\Time} \\
            \midrule
            1 & 10 & 2 & 30 & 10 & 3.0 & 0.23 \\
            1 & 10 & 3 & 30 & 10 & 3.0 & 0.17 \\
            2 & 10 & 2 & 30 & 10 & 3.0 & 0.40 \\
            2 & 10 & 3 & 30 & 10 & 3.0 & 0.31 \\
            1 & 20 & 2 & 60 & 20 & 3.0 & 600.0 \\
            1 & 20 & 3 & 60 & 20 & 3.0 & 17.89 \\
            2 & 20 & 2 & 60 & 20 & 3.0 & 1.40 \\
            2 & 20 & 3 & 60 & 20 & 3.0 & 3.94 \\
            1 & 40 & 2 & 96 & 40 & 2.4 & 600.0 \\
            1 & 40 & 3 & 111 & 40 & 2.77 & 600.0 \\
            2 & 40 & 2 & 100 & 40 & 2.5 & 600.0 \\
            2 & 40 & 3 & 111 & 40 & 2.77 & 600.0 \\
            \bottomrule
        \end{tabular}%
    }
\end{table}

\subsubsection{Computational Performance and Solution Quality}
The model was evaluated across three dimensions: patient count ($P \in \{10, 20, 40\}$), number of physicians ($\{1, 2\}$), and operating room blocks ($\{2, 3\}$). For each configuration, up to three feasible time slots per patient were generated, yielding 30--111 total candidate events (Table \ref{tab:benchmark_data}).

Key findings: (1) For operational scales ($P \le 20$), the solver delivers solutions in under 18 seconds, enabling real-time consultation-integrated workflow. (2) Increasing OR availability from 2 to 3 blocks improves runtime from 600s (timeout) to ~18s for $P=20$ (Figure \ref{fig:time_uses}). (3) For larger instances ($P=40$) reaching the 600-second timeout, solutions remain feasible with mean optimality gaps below 15\% and sufficient option generation (2.4--2.77 options per patient; Figure \ref{fig:events_per_patient}). These scenarios are infrequent in practice due to the continuous-flow model; 600 seconds is acceptable for offline batch processing. (4) For operational scales ($P \le 20$), proven optimal or near-optimal solutions ensure high-quality scheduling decisions.

\subsubsection{Patient Confirmation and Iterative Scheduling}

The framework's second phase guides patient review and slot confirmation. The MILP generates up to three feasible options per patient; using availability score $\sigma_p$, the system identifies flexible patients for early confirmation, then recomputes remaining options. This iterative approach ensures flexible patients consume non-critical options first, preserving choice for constrained patients. Patients review validated options ranked by clinical desirability, enabling shared decision-making that considers personal and occupational constraints.

\subsubsection{Scalability Perspective}
The computational performance aligns with the system's consultation-driven design. Instances with $P \le 20$ solve optimally in real-time; larger instances ($P = 40$) reach time limits but produce solutions with 15\% optimality gap for infrequent batch scenarios. The limited event pool (3 options per patient) maintains tractability while preserving patient choice. The MILP formulation balances solution quality, efficiency, and deployability for the continuous-flow paradigm.

\begin{figure}[ht]
    \centering
    \begingroup
        \includegraphics[scale=0.38]{figures/test_simulation_v1.pdf}
    \endgroup
    \caption{Average slots per scheduled patient}
    \label{fig:events_per_patient}
\end{figure}

\begin{figure}[ht]
    \centering
    \includegraphics[width=1\columnwidth]{figures/time_uses.pdf}
    \caption{Run time per instance (line plot with logarithmic scale)}
    \label{fig:time_uses}
\end{figure}

