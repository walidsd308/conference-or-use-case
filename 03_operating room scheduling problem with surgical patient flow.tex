\section{Operating Room Scheduling Problem with Surgical Patient Flow}
\begin{table*}[ht]
\centering
\begin{tabular*}{\textwidth}{@{\extracolsep{\fill}}ll}
\toprule
\textbf{Symbol} & \textbf{Description} \\ 
\midrule
$\mathcal{D}$ & Set of planning days, index $d$ \\
$\mathcal{B}$ & Set of operating rooms (blocs), index $b$ \\
$\mathcal{H}$ & Set of physicians, index $h$ \\
$\mathcal{P}$ & Set of patients, index $p$ \\
$\mathcal{I} = \{0,\dots,I_{\max}-1\}$ & Set of proposed time slot option indices for each patient, index $i$ \\
$\mathcal{K}$ & Set of operation types, index $k$ \\
$\mathcal{D}^{\text{part}}$ & Family of day partitions (for spacing constraint), each partition $P \in \mathcal{D}^{\text{part}}$ is a subset of $\mathcal{D}$ \\
\midrule
$\tau_p$ & Operating time of patient $p$ \\
$\tau^{\text{post}}_p$ & Post-operative time of patient $p$\\
$\pi_p$ & Priority of patient $p$ \\
$k(p)\in\mathcal{K}$ & Operation type of patient $p$\\
$d^{\text{arr}}_p$ & Earliest feasible day index for patient $p$\\
$d^{\text{dep}}_p$ & Latest feasible day index for patient $p$ \\
$T$ & Planning horizon length (in days); $\mathcal{D} = \{0,\ldots,T-1\}$\\
$\text{stdPost}$ & Standard post-operative buffer time (scheduling flexibility allowance added to block capacity) \\
$B_{b,d} \in \{0,1\}$ & Indicator: block $b$ is open on day $d$ (when $B_{b,d}=0$, capacity is zero)\\
$C_{b,d}$ & Available OR-time on $(b,d)$\\
$\overline{E}_{b,d}$ & Upper bound on capacity violation slack \\
$w_{p,h}$ & Compatibility weight between patient $p$ and physician $h$\\
$a_{p,h,b,d}$ & Preference weight for assigning $(p,h)$ to $(b,d)$\\
$n^{\text{pre}}_{b,d}$ & Number of pre-defined operation types in block $b$ on day $d$ (aggregated across all physicians)\\
$M$ & Big-M constant, e.g.\ $M \geq|\mathcal{B}|\,|\mathcal{H}|\,|\mathcal{P}|\,|\mathcal{I}|$\\
\midrule
$\alpha_1=10^3$ & Weight of patient-assignment errors \\
$\alpha_2=10^1$ & The weight of the block capacity violation (\scriptsize{is only used to schedule urgency, replacing elective patients)}.\\
$\alpha_3=10^1$ & Weight of early/late scheduling w.r.t.\ stay \\
$\alpha_4=10^{-1}$ & Weight of specialty fragmentation \\
$\alpha_5=10^{-3}$ & Weight of day usage \\
\bottomrule
\end{tabular*}
\caption{Data description (sets and parameters).}
\label{tab:parameters}
\end{table*}

\begin{table*}[ht]
\centering
\begin{tabular*}{\textwidth}{@{\extracolsep{\fill}}lll}
\toprule
\textbf{Variable} & \textbf{Domain} & \textbf{Description} \\ 
\midrule
$X_{d,b,h,p,i}$ & $\{0,1\}$ &
1 if patient $p$ is scheduled on day $d$ in block $b$ with physician $h$ and index $i$ \\
$z_d$ & $\{0,1\}$ &
1 if day $d$ is used (at least one assignment), 0 otherwise \\
$e_{p,i}$ & $\{0,1\}$ &
1 if patient $p$ is not assigned at index $i$ (slack for one-time assignment) \\
$E_{b,d}$ & $\mathbb{Z}_{\ge 0}$ &
Capacity violation (extra time) allowed for block $b$ on day $d$ \\
$EA_{p,i}$ & $\{0,1\}$ &
Early-arrival violation indicator for patient $p$ at index $i$ \\
$LD_{p,i}$ & $\{0,1\}$ &
Late-departure violation indicator for patient $p$ at index $i$ \\
$N_{b,h}$ & $\mathbb{Z}_{\ge 0}$ &
Upper bound on the number of distinct operation types for physician $h$ in block $b$ on any single day \\
$u_{d,b,h,k}$ & $\{0,1\}$ &
1 if physician $h$ performs operation type $k$ in block $b$ on day $d$ \\
\bottomrule
\end{tabular*}
\caption{Decision variables.}
\label{tab:variables}
\end{table*}

\subsection{Problem Statement}

The problem addresses the medium-term scheduling of elective orthopedic surgery patients at CHU Montpellier. This task generates an optimized schedule by assigning patients ($p \in \mathcal{P}$) to specific days ($d \in \mathcal{D}$), operating rooms ($b \in \mathcal{B}$), and physicians ($h \in \mathcal{H}$), within the planning horizon $T$ (see Table \ref{tab:parameters} for notation).

We formulate the scheduling decision as an optimization problem satisfying multiple constraints and optimizing a hierarchical multi-criteria objective function:


\textbf{Clinical and Temporal Constraints:} The solution must respect essential patient-specific requirements, including the operating time ($\tau_p$), post-operative time ($\tau^{\text{post}}_p$), and the patient's earliest ($d^{\text{arr}}_p$) and latest ($d^{\text{dep}}_p$) feasible scheduling days. Furthermore, the schedule must adhere to the resource availability (e.g., OR opening indicator $B_{b,d}$ and capacity $C_{b,d}$) and the compatibility profiles ($w_{p,h}$) between patients and physicians.
    
\textbf{Institutional Constraints:} The model enforces block allocation rules by minimizing unplanned overtime (represented by the capacity violation slack variable $E_{b,d}$) and respecting pre-defined assignments ($n^{\text{pre}}_{b,d}$).
    
\textbf{Hierarchical Optimization Objectives:} The objective minimizes a weighted sum of violations, structured hierarchically by penalty weights ($\alpha_1$ to $\alpha_5$):
    \begin{itemize}
        \item \textbf{Priority 1 (Highest):} Minimize non-feasible patient assignments (driven by $\alpha_1$).
        \item \textbf{Priority 2:} Minimize capacity and time constraint violations (driven by $\alpha_2$) and minimize early or late scheduling relative to the patient's stay limits (driven by $\alpha_3$).
        \item \textbf{Priority 3 (Lowest):} Minimize specialty fragmentation (driven by $\alpha_4$) and efficiently manage day usage (driven by $\alpha_5$), which promotes operational stability and maximizes resource utilization.
    \end{itemize}


\subsection{Operational Context and Patient Flow}

The scheduling framework operates within a specific clinical workflow that differs from traditional batch scheduling approaches. In practice, patients are not accumulated and scheduled in large batches at fixed weekly intervals. Instead, the system processes patients incrementally following their consultation appointments, where the operating physician determines the surgical need and assigns the patient to their own surgical schedule.

In this incremental process:
\begin{itemize}
    \item Patients enter the scheduling system immediately after consultation.
    \item The optimization model runs for each new patient (or small groups of patients from a single consultation day).
    \item The system generates up to three optimized feasible time slots tailored to each patient's constraints.
    \item Patients review these options and confirm their preferred surgery date.
    \item Once confirmed, the assignment becomes fixed in the schedule for subsequent optimization runs.
\end{itemize}

Consequently, each physician maintains a wait list of patients pending surgical scheduling, which typically contains between 5 and 15 patients at any given time. This wait-list represents patients who have been consulted and are awaiting either (a) the generation of feasible time slots, (b) patient confirmation of their selected slot, or (c) resolution of specific medical or administrative prerequisites. 

Wait lists remain small (5-15 patients, maximum 20) because patients flow continuously through the system rather than accumulating into backlogs.

Section \ref{sec:numerical_results} uses instance sizes of 10-40 patients, matching the typical daily to weekly scheduling workload.
\subsection{Mathematical model}

The operational scheduling problem is formulated as a Mixed-Integer Linear Program (MILP), which seeks to optimize patient assignments across operating rooms and days. The model addresses multiple conflicting objectives with hierarchical priorities (reflected by the $\alpha$ weights in Table \ref{tab:parameters}). This guarantees that higher-priority goals, such as minimizing assignment errors ($\alpha_1=10^3$), are satisfied before addressing lower-priority concerns. The model's sets and parameters are detailed in Table \ref{tab:parameters}, and its decision variables are defined in Table \ref{tab:variables}.

\subsubsection{Objective}
The objective function, $\max Z$, is a multi-objective optimization problem with a layered hierarchy enforced by hierarchically ordered weights ($\alpha_k$). The function maximizes the reward component for high-priority assignments ($\pi_p$) and preferred assignments (based on $w_{p,h}$ and $a_{p,h,b,d}$), where the reward is proportionally greater for assignments scheduled earlier in the block (smaller intra-day index $i$). The model balances rewards against penalties for assignment errors ($\alpha_1$, $e_{p,i}$), capacity violations ($\alpha_2$, $E_{b,d}$), early/late patient scheduling ($\alpha_3$, $EA_{p,i}$, $LD_{p,i}$), and specialty fragmentation ($\alpha_4$, $N_{b,h}$).

\begin{align}
\max \; Z
=&\; +\sum_{d,b,h,p,i}X_{d,b,h,p,i}\,\Big[ \pi_p (T-d) + w_{p,h} \nonumber \label{obj:priority-reward}\\
& \qquad + a_{p,h,b,d} \Big] (i+1)^{-1}\\
&\; - \alpha_1\sum_{p,i}\pi_p \,(i+1)^{-5} \, e_{p,i} \label{obj:error-patient} \\
&\; - \alpha_2\sum_{b,d}E_{b,d} \label{obj:error-bloc-day} \\
&\; - \alpha_3\sum_{p,i}\left( EA_{p,i} + LD_{p,i} \right) \label{obj:stay-violations} \\
&\; - \alpha_4\sum_{b,h}N_{b,h} \label{obj:specialty-fragmentation} \\
&\; - \alpha_5\sum_{d} d \, z_d \label{obj:day-usage}
\end{align}



\subsubsection{Constraints}
\textbf{Patient one-time assignment (with slack).}

\begin{equation}
\sum_{d\in\mathcal{D}} \sum_{b\in\mathcal{B}} \sum_{h\in\mathcal{H}}
   X_{d,b,h,p,i}
\;+\;
e_{p,i}
= 1,
\qquad
\forall p\in\mathcal{P},\; \forall i\in\mathcal{I}.
\label{c:one-assignment}
\end{equation}
(If $e_{p,i}=1$ then $p$ is not assigned at index $i$; otherwise $p$ must be assigned exactly once at that index.)

\textbf{Daily block capacity constraints (with slack).}

\begin{align}
&\sum_{h\in\mathcal{H}} \sum_{p\in\mathcal{P}} \sum_{i\in\mathcal{I}}
   X_{d,b,h,p,i}\, (\tau_p + \tau^{\text{post}}_p)
\;\le\;
C_{b,d} + \text{stdPost} + E_{b,d},
\label{c:capacity}\\
&0 \le E_{b,d} \le \overline{E}_{b,d},
\qquad \forall b\in\mathcal{B},\; \forall d\in\mathcal{D}: B_{b,d}=1.
\end{align}

\textbf{Patient availability vs.\ arrival/departure (early/late violations).}

Let $d^{\text{arr}}_p$ and $d^{\text{dep}}_p$ be the first/last admissible day indices for patient $p$.

\begin{equation}
\sum_{\substack{d\in\mathcal{D}:\\ d < d^{\text{arr}}_p}}
\;\sum_{b\in\mathcal{B}}\sum_{h\in\mathcal{H}}
X_{d,b,h,p,i}
= EA_{p,i},
\quad \forall p\in\mathcal{P},\; \forall i\in\mathcal{I}
\label{c:early}
\end{equation}
($EA_{p,i}$ may be fixed to $0$ if early reprogramming is not allowed for $p$).

\begin{equation}
\sum_{\substack{d\in\mathcal{D}:\\ d > d^{\text{dep}}_p}}
\;\sum_{b\in\mathcal{B}}\sum_{h\in\mathcal{H}}
X_{d,b,h,p,i}
= LD_{p,i},
\quad \forall p\in\mathcal{P},\; \forall i\in\mathcal{I}
\label{c:late}
\end{equation}
($LD_{p,i}$ is fixed to $0$ if late reprogramming is not allowed for $p$).

\textbf{Day-activation constraints.}

\begin{equation}
\sum_{b\in\mathcal{B}}\sum_{h\in\mathcal{H}}\sum_{p\in\mathcal{P}}\sum_{i\in\mathcal{I}}
X_{d,b,h,p,i}
\;\le\;
M \, z_d,
\qquad \forall d\in\mathcal{D}.
\label{c:day-activation}
\end{equation}
If any $X_{d,b,h,p,i}=1$ for a given $d$, then $z_d$ must be $1$ (and the day is penalized in the objective by \eqref{obj:day-usage}).

\textbf{Specialty indicators and fragmentation per doctor and room.}

\begin{align}
\forall p\in\mathcal{P}:& k(p)=k,\; \forall i\in\mathcal{I},\\
u_{d,b,h,k} &\ge X_{d,b,h,p,i},
\label{c:u-lower}\\
u_{d,b,h,k} &\le \sum_{p\in\mathcal{P}:k(p)=k}\sum_{i\in\mathcal{I}} X_{d,b,h,p,i},
\label{c:u-upper}
\end{align}
so $u_{d,b,h,k}$ becomes 1 if $(b,h)$ actually executes at least one case of type $k$ on day $d$.

The aggregated number of specialties per physician and block is constrained by
\begin{equation}
\sum_{k\in\mathcal{K}} u_{d,b,h,k} - 1 + n^{\text{pre}}_{b,d} \;\le\; N_{b,h},
 \forall d\in\mathcal{D},\; \forall b\in\mathcal{B},\; \forall h\in\mathcal{H}.
\label{c:N-bh}
\end{equation}

\textbf{Minimum spacing of appointments per patient and physician.}

\begin{equation}
\sum_{d\in P}\sum_{b\in\mathcal{B}}\sum_{i\in\mathcal{I}}
X_{d,b,h,p,i}
\;\le\; 1,
\qquad
\forall P\in\mathcal{D}^{\text{part}},\; \forall h\in\mathcal{H},\; \forall p\in\mathcal{P}.
\label{c:spacing}
\end{equation}
With partition interval equal to $1$ in the code, this effectively prevents multiple appointments of the same $(p,h)$ within the same partition interval.

\subsection{Flexibility-First Heuristic for Iterative Scheduling}
\label{sec:patient_ordering}

In clinical practice, patients select their time slots sequentially rather than in parallel. Once a patient confirms a slot, it becomes fixed for subsequent optimizations, reducing the option pool for remaining patients. The order in which patients are presented for selection becomes critical to maximizing feasibility and preserving choice for all patients.

This heuristic prioritizes patients with the most flexible schedules (high availability scores) for early selection. Such flexible patients can choose from multiple options without constraining remaining slots, thereby preserving options for constrained patients. For each unscheduled patient $p$, an availability score $\sigma_p$ quantifies scheduling flexibility:

\begin{equation}
\sigma_p = \sum_{s \in \mathcal{S}_p} \text{score}(s, p),
\end{equation}

where $\mathcal{S}_p$ is the set of feasible slots and $\text{score}(s, p)$ is:

\begin{equation}
\text{score}(s, p) = \begin{cases}
+100 & \text{if } s \in [d^{\text{arr}}_p, d^{\text{dep}}_p] \\
-|s - d^{\text{arr}}_p| & \text{if } s < d^{\text{arr}}_p \\
-|s - d^{\text{dep}}_p| & \text{if } s > d^{\text{dep}}_p
\end{cases}
\end{equation}

Patients with higher $\sigma_p$ are processed first. After each patient confirms a selection, slot availability is updated and scores are recomputed for remaining patients.

\begin{algorithm}[ht]
\caption{Iterative Patient Selection and Confirmation}
\label{alg:patient_ordering}
\begin{algorithmic}[1]
\State \textbf{Input:} Proposed slots $\mathcal{S}_{p}$ for each patient $p \in \mathcal{P}$, current assignments $A \gets \emptyset$
\State \textbf{Output:} Confirmed assignments $A^*$
\State Initialize $A^* \gets \emptyset$, $\mathcal{P}^{\text{remain}} \gets \mathcal{P}$
\While{$\mathcal{P}^{\text{remain}} \neq \emptyset$}
    \For{each patient $p \in \mathcal{P}^{\text{remain}}$}
        \State Compute score: $\sigma_p \gets \sum_{s \in \mathcal{S}_p} \text{score}(s, p)$
    \EndFor
    \State $p^* \gets \arg\max_p \sigma_p$ \quad \text{(select most flexible patient)}
    \State Sort $\mathcal{S}_{p^*}$ by $\text{score}(s, p^*)$ in descending order
    \State Display sorted slots to user (physician or patient)
    \State Receive user confirmation: $s^* \in \mathcal{S}_{p^*}$
    \State $A^* \gets A^* \cup \{(p^*, s^*)\}$
    \State $\mathcal{P}^{\text{remain}} \gets \mathcal{P}^{\text{remain}} \setminus \{p^*\}$
    \State Re-optimize: Update $\mathcal{S}_p$ for all $p \in \mathcal{P}^{\text{remain}}$ given fixed assignments $A^*$
\EndWhile
\State \Return $A^*$
\end{algorithmic}
\end{algorithm}


