\section{Related Work}

Operating room scheduling (ORS) has received substantial attention due to its complexity, operational cost, and direct impact on patient care. A recent comprehensive review by \cite{AlAmin2024} surveyed 260 studies published from 2000 to 2023, highlighting key trends such as a steep rise in optimization-based ORS, the dominance of mathematical programming approaches, and the increasing adoption of machine learning techniques for surgery duration prediction. The body of OR scheduling literature can be broadly classified into two strategic levels: \textit{tactical}, addressing medium-term master surgical schedules (weekly or monthly block allocation to specialties), and \textit{operational}, addressing short-term patient-to-slot assignments within fixed block structures. Within these levels, researchers employ deterministic models for predictable surgical workflows, stochastic models for uncertainty in durations and patient arrivals, and increasingly, machine learning approaches for data-driven decision support.

Recent advances have introduced several competing approaches to OR scheduling. Artificial intelligence-based methods \cite{Zhao2025,Eshghali2024,Lex2025,ElBalka2025} integrate machine learning with optimization, using reinforcement learning and predictive models to achieve 15–560 minute improvements in waiting times and utilization. Stochastic programming approaches \cite{Zhang2021,Miao2021,Tsang2025,Wang2024,Makboul2025} address uncertainty through MDPs, chance-constrained optimization, and distributional robustness, enabling dynamic re-optimization under risk-averse objectives. Downstream resource integration \cite{Schneider2020,MHallah2019,Celik2023} explicitly couples ICU/ward occupancy with OR scheduling, improving bed allocation and preventing infeasibilities. Advanced computational techniques \cite{Almoghrabi2025,Akbarzadeh2025,Bernardelli2024,Nash2025,Kamran2020} employ branch-price-and-cut, genetic algorithms, and game-theoretic concepts to handle large-scale combinatorial complexity and rapid re-optimization.

Notably, the review emphasizes persistent gaps such as limited attention to outpatient surgery and integrated intensive care unit (ICU) or ward constraints, motivating research into real-time, AI-enabled, and sustainability-aware frameworks. Beyond operational optimization, a critical and understudied gap persists: the absence of patient-centered scheduling frameworks that balance institutional efficiency with patient autonomy and shared decision-making. Existing literature predominantly treats patients as resources to be optimized rather than stakeholders whose preferences and comfort contribute meaningfully to surgical outcomes. This gap motivates the integration of patient choice into operational scheduling models a shift toward human-centered healthcare operations.

Among most directly relevant studies, \cite{Spratt2019} developed a real-time reactive framework addressing the Surgical Case Sequencing Problem with millisecond rescheduling, managing surgeon and OR specificity, short-notice elective add-ons, and disruptions. However, this approach lacks optimality guarantees and overlooks downstream ICU constraints. \cite{IntegratedRobust2025} proposed a complementary framework for hospital-based ambulatory surgery integration, addressing a critical gap where ambulatory-inpatient coordination remains understudied.

These limitations and emerging research directions strongly motivate our contribution, which directly addresses these gaps by proposing a patient-centered OR scheduling framework that (1) generates multiple distinct, feasible, and optimized time slot options for each patient, enabling genuine shared decision-making; (2) formulates the problem as a MILP with hierarchical objectives that enforce hard capacity constraints and treat overtime as an infeasibility-driven slack variable; and (3) has been deployed and validated in a real clinical environment, demonstrating that operational efficiency and patient autonomy are complementary rather than competing objectives.

