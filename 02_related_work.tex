\section{Related Work}

Operating room scheduling (ORS) has received substantial attention due to its complexity, operational cost, and direct impact on patient care. A recent comprehensive review by \cite{AlAmin2024} surveyed 260 studies published from 2000 to 2023, highlighting key trends such as a steep rise in optimization-based ORS, the dominance of mathematical programming approaches, and the increasing adoption of machine learning techniques for surgery duration prediction. The body of OR scheduling literature can be broadly classified into two strategic levels: tactical, addressing medium-term master surgical schedules (weekly or monthly block allocation to specialties), and operational, addressing short-term patient-to-slot assignments within fixed block structures. Within these levels, researchers employ deterministic models for predictable surgical workflows, stochastic models for uncertainty in durations and patient arrivals, and increasingly, machine learning approaches for data-driven decision support.

Recent advances have introduced several competing approaches to OR scheduling. Artificial intelligence-based methods \cite{Zhao2025,Eshghali2024,Lex2025,ElBalka2025} integrate machine learning with optimization, using reinforcement learning and predictive models to achieve 15–560 minute improvements in waiting times and utilization. Stochastic programming approaches \cite{Zhang2021,Miao2021,Tsang2025,Wang2024,Makboul2025} address uncertainty through MDPs, chance-constrained optimization, and distributional robustness, enabling dynamic re-optimization under risk-averse objectives. Downstream resource integration \cite{Schneider2020,MHallah2019,Celik2023} explicitly couples ICU/ward occupancy with OR scheduling, improving bed allocation and preventing infeasibilities. Advanced computational techniques \cite{Almoghrabi2025,Akbarzadeh2025,Bernardelli2024,Nash2025,Kamran2020} employ branch-price-and-cut, genetic algorithms, and game-theoretic concepts to handle large-scale combinatorial complexity and rapid re-optimization.

Notably, the review emphasizes persistent gaps such as limited attention to outpatient surgery and integrated intensive care unit (ICU) or ward constraints, motivating research into real-time, AI-enabled, and sustainability-aware frameworks. Beyond operational optimization, a critical and understudied gap persists: the absence of patient-centered scheduling frameworks that balance institutional efficiency with patient autonomy and shared decision-making. Existing literature predominantly treats patients as resources to be optimized rather than stakeholders whose preferences and comfort contribute meaningfully to surgical outcomes. This gap motivates the integration of patient choice into operational scheduling models a shift toward human-centered healthcare operations.

Among most directly relevant studies, \cite{Spratt2019} developed a real-time reactive framework addressing the Surgical Case Sequencing Problem, managing surgeon and OR specificity, short-notice elective add-ons, and disruptions through heuristic-based rapid rescheduling. While achieving practical real-time performance, this heuristic approach does not provide mathematical optimality guarantees and does not incorporate downstream bed resource constraints (ICU or ward occupancy) into the scheduling decisions. \cite{IntegratedRobust2025} proposed a complementary framework for hospital-based ambulatory surgery integration through distributionally robust optimization, addressing a critical gap where ambulatory-inpatient coordination remains understudied, though this work focuses on batch scheduling rather than continuous patient flow.

Critically, existing approaches—whether reactive heuristics, stochastic programming, or AI-based methods—predominantly generate a single schedule assignment per planning cycle, requiring complete rescheduling when patient preferences conflict with the proposed allocation. These limitations and emerging research directions strongly motivate our contribution, which proposes a fundamentally different continuous optimization paradigm for patient-centered OR scheduling. Our solution operates in two complementary phases: Phase 1 employs a Mixed-Integer Linear Program (MILP) that simultaneously considers all clinical and operational constraints (surgeon availability, OR capacity, patient arrival/departure windows, bed resources) to generate up to three mathematically optimal, distinct feasible time slot options for each patient. Phase 2 implements a flexibility-first iterative selection heuristic where patients with highest scheduling flexibility select and confirm their preferred option in consultation with their physician, after which the system immediately re-optimizes for remaining patients using updated resource availability. This approach ensures that (1) all generated options satisfy hard constraints with optimality guarantees from the MILP solver; (2) patient preferences are integrated during the scheduling process rather than imposed afterward; (3) the system maintains continuous responsiveness to confirmations through rapid re-optimization cycles (sub-18 second runtime for typical patient cohorts of 10–20 patients); and (4) operational efficiency and patient autonomy function as complementary rather than competing objectives, as validated through real clinical deployment at University Hospital Montpellier.

