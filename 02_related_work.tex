\section{Related Work}

Operating room scheduling (ORS) attracts research attention due to its complexity, operational cost, and impact on patient care. A recent review by \cite{AlAmin2024} surveyed 260 studies published from 2000 to 2023, highlighting key trends such as a steep rise in optimization-based ORS, the dominance of mathematical programming approaches, and the adoption of machine learning techniques for surgery duration prediction. OR scheduling literature divides into two strategic levels: tactical, addressing medium-term master surgical schedules (weekly or monthly block allocation to specialties), and operational, addressing short-term patient-to-slot assignments within fixed block structures. Within these levels, researchers employ deterministic models for predictable surgical workflows, stochastic models for uncertainty in durations and patient arrivals, and increasingly, machine learning approaches for data-driven decision support.

Recent advances have introduced multiple approaches to OR scheduling. Artificial intelligence-based methods \cite{Zhao2025,Eshghali2024,Lex2025,ElBalka2025} integrate machine learning with optimization, using reinforcement learning and predictive models to achieve 15–560 minute improvements in waiting times and utilization. Stochastic programming approaches \cite{Zhang2021,Miao2021,Tsang2025,Wang2024,Makboul2025} address uncertainty through MDPs, chance-constrained optimization, and distributional robustness, enabling dynamic re-optimization under risk-averse objectives. Downstream resource integration \cite{Schneider2020,MHallah2019,Celik2023} explicitly couples ICU/ward occupancy with OR scheduling, improving bed allocation and preventing infeasibilities. Advanced computational techniques \cite{Almoghrabi2025,Akbarzadeh2025,Bernardelli2024,Nash2025,Kamran2020} use branch-price-and-cut, genetic algorithms, and game-theoretic concepts to handle large-scale combinatorial complexity and rapid re-optimization.

Notably, the review emphasizes persistent gaps such as limited attention to outpatient surgery and integrated intensive care unit (ICU) or ward constraints, motivating research into real-time, AI-enabled, and sustainability-aware frameworks. Beyond operational optimization, an understudied gap persists: patient-centered scheduling frameworks that balance institutional efficiency with patient autonomy and shared decision-making remain rare. Existing literature treats patients as resources rather than stakeholders whose preferences and comfort contribute to surgical outcomes. This gap motivates integrating patient choice into scheduling models, shifting toward human-centered healthcare operations.

Among directly relevant studies, \cite{Spratt2019} developed a real-time reactive framework addressing the Surgical Case Sequencing Problem, managing surgeon and OR specificity, short-notice elective add-ons, and disruptions through heuristic-based rapid rescheduling. This heuristic approach achieves real-time performance but does not guarantee mathematical optimality or incorporate downstream bed resource constraints (ICU or ward occupancy) into the scheduling decisions. \cite{IntegratedRobust2025} proposed a complementary framework for hospital-based ambulatory surgery integration through distributionally robust optimization, addressing a critical gap where ambulatory-inpatient coordination remains understudied, though this work focuses on batch scheduling rather than continuous patient flow.

Existing approaches generate a single schedule assignment per planning cycle. This work proposes a continuous optimization paradigm where a MILP generates up to three feasible time slot options for each patient considering all clinical and operational constraints. Patients select their preferred option in consultation with their physician, and the system immediately re-optimizes for remaining patients.

