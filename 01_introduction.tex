\section{Introduction}
Operating room (OR) scheduling is formally recognized as a complex, multi-objective optimization problem within healthcare operations management. OR scheduling directly impacts healthcare delivery, influencing operational costs, resource utilization, and crucially, patient experience and clinical outcomes. Rising healthcare expenditures, limited surgical resources, and growing patient expectations for timely access to care have made OR scheduling increasingly critical. While conventional approaches prioritize operational metrics maximizing OR utilization, minimizing overtime, and reducing waiting times through deterministic or stochastic optimization models they typically impose predetermined schedules without incorporating patient preferences or autonomy. This operational-centric perspective, while efficient in resource management, often overlooks the human dimension of surgical care, including patient anxiety, schedule disruption costs, and the therapeutic value of patient involvement in treatment planning.

This study is set in the orthopedic department at the University Hospital of Montpellier (France), which provides an ideal environment for developing a patient-centered scheduling approach. Unlike emergency or time-critical surgical cases, elective orthopedic patients typically have more flexible scheduling preferences, even when urgent cases are involved. This flexibility creates a unique opportunity to integrate the human dimension into the scheduling decision-making process, moving beyond purely operational optimization.
This work balances operational efficiency with patient care, enabling patients to actively participate in the scheduling process.

\begin{figure}[ht]
    \centering
    \includegraphics[scale=0.38]{figures/model_flow_diagram.pdf}
    \caption{Iterative scheduling process: MILP generates feasible events, Flexibility-First heuristic selects and confirms patient selections, and unused events are redistributed iteratively.}

    \label{fig:methodology_flow}
\end{figure}

This approach is implemented through an iterative optimization methodology, illustrated in Figure \ref{fig:methodology_flow}. The MILP model generates up to three distinct feasible time slots for each patient, with each option satisfying all clinical and operational constraints. The Flexibility-First heuristic identifies the patient with the highest scheduling flexibility, measured by availability score $\sigma_p$ (detailed in Subsection \ref{sec:patient_ordering}). The selected patient reviews the available options and confirms one preferred event. The system then redistributes unused events to remaining patients and recalculates the next patient to process. This approach transforms patients from passive recipients of scheduling decisions into active participants in the planning process, demonstrating that patient autonomy and operational efficiency are complementary objectives.

