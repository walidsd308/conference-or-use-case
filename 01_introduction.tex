\section{Introduction}
Operating room (OR) scheduling is formally recognized as a complex, multi-objective optimization problem within healthcare operations management. OR scheduling directly impacts healthcare delivery, influencing operational costs, resource utilization, and crucially, patient experience and clinical outcomes. Rising healthcare expenditures, limited surgical resources, and growing patient expectations for timely access to care have made OR scheduling increasingly critical. While conventional approaches prioritize operational metrics maximizing OR utilization, minimizing overtime, and reducing waiting times through deterministic or stochastic optimization models they typically impose predetermined schedules without incorporating patient preferences or autonomy. This operational-centric perspective, while efficient in resource management, often overlooks the human dimension of surgical care, including patient anxiety, schedule disruption costs, and the therapeutic value of patient involvement in treatment planning.

This study is set in the orthopedic department at the University Hospital of Montpellier (France), which provides an ideal environment for developing a patient-centered scheduling approach. Unlike emergency or time-critical surgical cases, elective orthopedic patients typically have more flexible scheduling preferences, even when urgent cases are involved. This flexibility creates a unique opportunity to integrate the human dimension into the scheduling decision-making process, moving beyond purely operational optimization.
\begin{figure}[ht]
    \centering
    \begingroup
    \sbox0{\includegraphics{figures/api_1.pdf}}%
        \includegraphics[clip,trim=0 0 0 0,width=9cm]{figures/api_1.pdf}
    \endgroup
    \caption{The main screen of the solution that connects with the API, showing the planning of the physician.}
    \label{fig:api_1}
\end{figure}

This work departs from conventional paradigms by balancing operational efficiency with patient-centered care, allowing patients to actively participate in the scheduling process to enhance both efficiency and experience.

\begin{figure}[ht]
    \centering
    \begingroup
    \sbox0{\includegraphics{figures/api_2.pdf}}%
        \includegraphics[clip,trim=0 0 0 0,width=8cm]{figures/api_2.pdf}
    \endgroup
    \caption{The last step of the event selection shows three optimized propositions of the model.}
    \label{fig:api_2}
\end{figure}
This patient-centered perspective is implemented through a new interaction model, as shown in figures \ref{fig:api_1} and \ref{fig:api_2}. Figure \ref{fig:api_1} shows the planning interface for physicians, which is integrated with the optimization API and displays the complete surgical schedule. Instead of imposing a rigid, predetermined schedule with a single time slot, the system generates multiple feasible scheduling options. As shown in Figure \ref{fig:api_2}, patients are presented with three distinct optimized options, each of which satisfies all clinical and operational constraints while offering genuine choice in their care timeline. This approach transforms patients from passive recipients of scheduling decisions into active participants in the planning process. This enhances patient engagement and satisfaction while maintaining operational efficiency and clinical safety.

